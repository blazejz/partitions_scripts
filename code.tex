\documentclass{article}

\usepackage[utf8]{inputenc}

\usepackage{geometry}
\usepackage{amsmath}
\usepackage{minted}

\setminted{
    linenos,
    mathescape,
    breaklines
}

\begin{document}

\section{$B_m$}

\begin{minted}{python}
def find_bm(m):
    # $\text{bound} := m^2 + m + 1$
    bound = m * m + m + 1
    # $\text{primes} := \text{set of primes from range } [m+2, \text{bound}+1)$
    primes = set(find_primes(m + 2, bound + 1))

    # results dictionary
    results = {}

    # initialize main table
    b = [1]
    # counter
    i = 1

    # while any prime remains
    while len(primes):
        # if $i \equiv 1 \pmod{m}$
        if i % m == 0:
            # $t := b_{\lfloor i / 5\rfloor} + b_{i - 5}$
            t = b[i // 5] + b[i - 5]
        else:
            # $t := b_{i - 1}$
            t = b[i - 1]
        b.append(t)
        # foreach prime $p$ remaining
        for p in primes.copy():
            # if $t \equiv 0 \pmod{p}$
            if t % p == 0:
                # remove $p$ from set
                primes.remove(p)
                # set result for the prime $p$
                results[p] = {'n': i, 'b': t}
        i += 1

    return results
\end{minted}

\clearpage
\section{$C_m$}

Analogically to $B_m$

\begin{minted}{python}
def find_cm(m):
    # $\text{bound} := m^2 + m + 1$
    bound = m * m + m + 1
    # $\text{primes} := \text{set of primes from range } [m+2, \text{bound}+1)$
    primes = set(find_primes(m + 2, bound + 1))

    # results dictionary
    results = {}

    # initialize main table
    b = [1, 1]
    # counter
    i = 2

    # while any prime remains
    while len(primes):
        # if $i \equiv 1 \pmod{m}$
        if i % m == 1:
            # $t := b_{\lfloor (i - 1) / 5\rfloor} + b_{i - 5}$
            t = b[(i - 1) // 5] + b[i - 5]
        else:
            # $t := b_{i - 1}$
            t = b[i - 1]
        b.append(t)

        # foreach prime $p$ remaining
        for p in primes.copy():
            # if $t \equiv 0 \pmod{p}$
            if t % p == 0:
                # remove $p$ from set
                primes.remove(p)
                # set result for the prime $p$
                results[p] = {'n': i, 'b': t}
        i += 1

    return results
\end{minted}

\clearpage
\section{SYF2}

Note that in computer science matrix notation is a bit different than in mathematics.  We use $A[i]$ to denote $i$-th row of array $A$ and $A[i][j]$ ad $j$-th cell of $i$-th row.

\subsection{Main method}

\begin{minted}{python}
# Main method
#
# arguments:
#   mod: modulo
#   levels: total number of generated levels
#   arity
#   step: step of algorithm
def main(mod, levels, arity, step):
    # list of precomputed levels
    levels_list = []

    # generate levels for given modulo and arity upto given depth
    for i, lvl in enumerate(gen_levels(mod, arity)):
        # store the level
        levels_list.append(lvl)
        # stop the process if we generated required number of levels
        if i >= levels - 1:
            break

    # check, whether generated case is valid
    result = check(0, None, mod, levels_list, step)

    # we count levels from zero, so add 1 to the result
    if result is not None:
        result += 1

    # print the result
    print(result)
\end{minted}

\subsection{Level generator}

\begin{minted}{python}
# Level Generator
#
# arguments:
#   mod: modulo
#   arity
def gen_levels(mod, arity):
    # level is initially 2-D array $[1]$
    old = np.array([[1]], dtype=TYPE)

    # yield the result keeping state of the generator
    yield old

    # infinite loop
    while True:
        # set old_h and old_w to height and width of previous level
        old_h, old_w = old.shape

        # compute new dimensions $\text{h} := 1 + \text{old\_h} \cdot \text{arity}$
        h = 1 + old_h * arity
        # $\text{w} := \text{old\_w} + 1$
        w = old_w + 1

        # initialize new level with zeros
        new = np.zeros((h, w), dtype=TYPE)
        # add $1$ to first column
        new[:, 0] += 1
        # for each $row$ in the array $i$ is number of the $row$
        for i, row in enumerate(old):
            # for $j$ in range $[0, arity)$
            for j in range(arity):
                # for each row in the new level with index $\ge 2 i + 1 + j$ add $[0, \text{row from the old level}]$
                new[2*i+1+j:, 1:] += row

        # apply modulo on new level
        new %= mod

        # we need to remember only the last level
        old = new
        # yield generated level
        yield old
\end{minted}

\subsection{Level checker}

\begin{minted}{python}
# Recursive Level Checker
#
# arguments:
#   lvl: currently beeing checked level
#   prev: previous table
#   mod: modulo
#   levels: total number of generated levels
#   step: step of algorithm
def check(lvl, prev, mod, levels, step):
    # if we checked the whole generated depth, return failure
    if lvl >= len(levels):
        return None

    # set $\text{result}$ initially to level number
    result = lvl

    if prev is None:
        # if we have no previous, define $\text{prev} := [0]$
        prev = np.zeros(1)
    else:
        # else add $1$-row padding from the left using value $0$
        prev = np.pad(prev, (1, 0), mode='constant')

    # for $k$ in $0, \text{step}, 2\text{step}, 3\text{step}, \ldots, \text{mod}-1$
    for k in range(0, mod, step):
        prev[0] = k
        # Let's multiply each row of considered level and multiply it elementwise by prev.  If sum of the received row is equivalent to $0$ modulo $mod$, continue the search
        if np.any(((levels[lvl] * prev).sum(axis=1) % mod) == 0):
            continue
        # the oposite case
        else:
            # check next level recursively
            r = check(lvl + 1, prev, mod, levels, step)
            if r is None:
                # if algorithm wasn't able to find satisfying result, return failure
                return None
            else:
                # in the oposite case, remember only the greatest result
                result = max(result, r)

    return result
\end{minted}

\end{document}
